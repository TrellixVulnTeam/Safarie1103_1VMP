% Options for packages loaded elsewhere
\PassOptionsToPackage{unicode}{hyperref}
\PassOptionsToPackage{hyphens}{url}
%
\documentclass[
]{article}
\usepackage{lmodern}
\usepackage{amssymb,amsmath}
\usepackage{ifxetex,ifluatex}
\ifnum 0\ifxetex 1\fi\ifluatex 1\fi=0 % if pdftex
  \usepackage[T1]{fontenc}
  \usepackage[utf8]{inputenc}
  \usepackage{textcomp} % provide euro and other symbols
\else % if luatex or xetex
  \usepackage{unicode-math}
  \defaultfontfeatures{Scale=MatchLowercase}
  \defaultfontfeatures[\rmfamily]{Ligatures=TeX,Scale=1}
\fi
% Use upquote if available, for straight quotes in verbatim environments
\IfFileExists{upquote.sty}{\usepackage{upquote}}{}
\IfFileExists{microtype.sty}{% use microtype if available
  \usepackage[]{microtype}
  \UseMicrotypeSet[protrusion]{basicmath} % disable protrusion for tt fonts
}{}
\makeatletter
\@ifundefined{KOMAClassName}{% if non-KOMA class
  \IfFileExists{parskip.sty}{%
    \usepackage{parskip}
  }{% else
    \setlength{\parindent}{0pt}
    \setlength{\parskip}{6pt plus 2pt minus 1pt}}
}{% if KOMA class
  \KOMAoptions{parskip=half}}
\makeatother
\usepackage{xcolor}
\IfFileExists{xurl.sty}{\usepackage{xurl}}{} % add URL line breaks if available
\IfFileExists{bookmark.sty}{\usepackage{bookmark}}{\usepackage{hyperref}}
\hypersetup{
  pdftitle={Assignment 9.2 - Clustering},
  pdfauthor={edris safari},
  hidelinks,
  pdfcreator={LaTeX via pandoc}}
\urlstyle{same} % disable monospaced font for URLs
\usepackage[margin=1in]{geometry}
\usepackage{graphicx,grffile}
\makeatletter
\def\maxwidth{\ifdim\Gin@nat@width>\linewidth\linewidth\else\Gin@nat@width\fi}
\def\maxheight{\ifdim\Gin@nat@height>\textheight\textheight\else\Gin@nat@height\fi}
\makeatother
% Scale images if necessary, so that they will not overflow the page
% margins by default, and it is still possible to overwrite the defaults
% using explicit options in \includegraphics[width, height, ...]{}
\setkeys{Gin}{width=\maxwidth,height=\maxheight,keepaspectratio}
% Set default figure placement to htbp
\makeatletter
\def\fps@figure{htbp}
\makeatother
\setlength{\emergencystretch}{3em} % prevent overfull lines
\providecommand{\tightlist}{%
  \setlength{\itemsep}{0pt}\setlength{\parskip}{0pt}}
\setcounter{secnumdepth}{-\maxdimen} % remove section numbering

\title{Assignment 9.2 - Clustering}
\author{edris safari}
\date{2/7/2020}

\begin{document}
\maketitle

\hypertarget{assignment-9.2---clustering}{%
\subsection{Assignment 9.2 -
Clustering}\label{assignment-9.2---clustering}}

Labeled data is not always available. For these types of datasets, you
can use unsupervised algorithms to extract structure. The k-means
clustering algorithm and the k nearest neighbor algorithm both use the
Euclidean distance between points to group data points. The difference
is the k-means clustering algorithm does not use labeled data.

In this problem, you will use the k-means clustering algorithm to look
for patterns in an unlabeled dataset. The dataset for this problem is
found at \url{clustering-data.csv}.

\begin{enumerate}
\def\labelenumi{\alph{enumi}.}
\item
  Plot the dataset using a scatter plot.
\item
  Fit the dataset using the k-means algorithm from k=2 to k=12. Create a
  scatter plot of the resultant clusters for each value of k.
\item
  As k-means is an unsupervised algorithm, you cannot compute the
  accuracy as there are no correct values to compare the output to.
  Instead, you will use the average distance from the center of each
  cluster as a measure of how well the model fits the data. To calculate
  this metric, simply compute the distance of each data point to the
  center of the cluster it is assigned to and take the average value of
  all of those distances.
\end{enumerate}

Calculate this average distance from the center of each cluster for each
value of k and plot it as a line chart where k is the x-axis and the
average distance is the y-axis.

\begin{enumerate}
\def\labelenumi{\alph{enumi}.}
\setcounter{enumi}{3}
\tightlist
\item
  One way of determining the ``right'' number of clusters is to look at
  the graph of k versus average distance and finding the ``elbow
  point''. Looking at the graph you generated in the previous example,
  what is the elbow point for this dataset?
\end{enumerate}

\hypertarget{import-dataset}{%
\subsubsection{Import dataset}\label{import-dataset}}

\begin{verbatim}
## [1] "C:/Users/safar/Documents/GitHub/Safarie1103/Bellevue University/Courses/DSC520/Week9"
\end{verbatim}

\begin{verbatim}
## [1] "C:/Users/safar/Documents/GitHub/Safarie1103/Bellevue University/Courses/DSC520/Week9"
\end{verbatim}

\begin{verbatim}
## 'data.frame':    4022 obs. of  2 variables:
##  $ x: int  46 69 144 171 194 195 221 244 45 47 ...
##  $ y: int  236 236 236 236 236 236 236 236 235 235 ...
\end{verbatim}

\begin{verbatim}
##     x   y
## 1  46 236
## 2  69 236
## 3 144 236
## 4 171 236
## 5 194 236
## 6 195 236
\end{verbatim}

\hypertarget{scatter-plots}{%
\subsubsection{Scatter plots}\label{scatter-plots}}

\hypertarget{scatter-plot-of-binary-data}{%
\paragraph{Scatter plot of binary
data}\label{scatter-plot-of-binary-data}}

\includegraphics{Assignment-9.3-Clustering_files/figure-latex/scatter plot clustering_data-1.pdf}

\includegraphics{Assignment-9.3-Clustering_files/figure-latex/2-1.pdf}

\includegraphics{Assignment-9.3-Clustering_files/figure-latex/4-1.pdf}

\end{document}
